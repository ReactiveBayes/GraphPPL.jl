
% JuliaCon proceedings template
\documentclass{juliacon}
\setcounter{page}{1}

\begin{document}

% **************GENERATED FILE, DO NOT EDIT**************

\title{GraphPPL.jl: A Julia package for probabilistic model and Bethe Free Energy optimization constraints specification}

\author[1]{Dmitry Bagaev}
\author[1]{Bert de Vries}
\affil[1]{Technical University of Eindhoven, BIASlab}

\keywords{probabilistic programming, Bayesian inference, variational inference, factor graphs, bethe free energy, graphical models, optimization}

\hypersetup{
pdftitle = {GraphPPL.jl: A Julia package for probabilistic model and Bethe Free Energy optimization constraints specification},
pdfsubject = {JuliaCon 2019 Proceedings},
pdfauthor = {Dmitry Bagaev, Bert de Vries},
pdfkeywords = {probabilistic programming, Bayesian inference, variational inference, factor graphs, bethe free energy, graphical models, optimization},
}



\maketitle

\begin{abstract}

GraphPPL.jl is a Julia package for the specification of a probabilistic model and constraints on the variational Bayesian inference procedure. 
The package defines a macro-based language for convenient specification of a probabilistic model. In addition, the package includes a macro-based language to specify extra factorization and functional form constraints on local variational distributions in different parts of the model's factor graph. These extra constraints enable efficient hybrid inference with a combination of different variational inference techniques in one model. 

\end{abstract}

\section{Background}

In order to apply Bayesian modeling framework, we first need to specify a so-called generative probabilistic model, which represents our beliefs about how the data we observe might have been generated. These models often contain latent states that cannot be observed directly. Bayesian inference then proceeds by marginalizing out all latent variables except those in which we are interested. Many useful probabilistic models, however, often contain a large number of latent variables and complex conditional dependencies between these variables. Analytic computation of the marginalization integrals for such large probabilistic models is, in many cases, unfeasible. As a result, exact Bayesian inference in many models of interest is, usually, intractable and requires the application of variational Bayesian methods. These methods proceed by minimization of some variational objective (e.g., a Kullback-Leibler (KL) divergence) that approximates a distance between the exact Bayesian solution and a variational distribution, which is constrained to be "simple" enough for inference to be tractable, and "general" enough to be as close as possible to the exact solution.

\section{Problem statement}

Most packages for automated Bayesian inference in the Julia language ecosystem, such as Turing.jl \cite{ge_turing_2018}, accept a model definition and provide a single method (usually Monte Carlo sampling-based) that must be used for inference. For many applications, in particular, for real-time inference a model that processes streaming data, the standard inference methods are not adequate. Conjugate relationships between random variables often enable the usage of analytical marginal computations without the need for an expensive sampling procedure. It would be more efficient to have a user-friendly constraints specification language that enables the application of different Bayesian inference methods in one model: belief propagation or variational structured/mean-field optimization in the conjugate parts of the model and sampling-based black-box methods in other parts.

\section{Solution proposal}

In this contribution, we present GraphPPL.jl, which is a Julia package for the definition of probabilistic models and the specification of variational Bayesian inference constraints. The package provides a user-friendly and comprehensive meta-language for the specification of both a probabilistic model and variational inference constraints that balance the accuracy of inference results with computational costs. GraphPPL.jl exports the \texttt{@model} macro to create a probabilistic model in the form of a factor graph that is compatible with ReactiveMP.jl's \cite{bagaev_reactivempjl_2022} reactive message passing-based inference engine. To enable fast and accurate inference, all message update rules default to pre-computed closed-form solutions. The ReactiveMP.jl package already implements a large selection of these pre-computed rules. If an analytical solution is not available, then the GraphPPL.jl package provides ways to tweak, relax, and customize local constraints in selected parts of the factor graph. To simplify this process, the package exports the \texttt{@constraints} macro to specify extra factorization and form constraints on the variational posterior \cite{senoz_variational_2021}. For advanced use cases, GraphPPL.jl exports the \texttt{@meta} macro that enables custom message passing inference modifications for each node in a factor graph representation of the model. This approach enables local approximation methods (e.g., sampling-based) only if necessary and allows for efficient variational Bayesian inference.

% Evaluation
\section{Evaluation}

Over the past two years, our probabilistic modeling ecosystem, comprising GraphPPL.jl and ReactiveMP.jl, has been battle-tested on many sophisticated models. These simulations have led to several publications in high-ranked journals such as Entropy \cite{podusenko_message_2021}, Frontiers \cite{podusenko_aida_2022} and Applied Sciences \cite{van_erp_bayesian_2021}, and conferences like MLSP-2021 \cite{podusenko_message_2021_mslp} and EUSIPCO-2022 \cite{podusenko_message_2022_eusipco}. The current contribution enables a user-friendly approach to large, complex and sophisticated Bayesian modeling problems. 

% **************GENERATED FILE, DO NOT EDIT**************

\bibliographystyle{juliacon}
\bibliography{ref.bib}


\end{document}

% Inspired by the International Journal of Computer Applications template
